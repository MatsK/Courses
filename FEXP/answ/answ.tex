\documentclass[11pt,a4,epsf]{jarticle}
%\usepackage{amssymb}
%% for apple LaserWriter Series %%
%% 
\setlength{\topmargin}{-0.5in}
\setlength{\textwidth}{5.6in}
\setlength{\textheight}{8.8in}
\setlength{\oddsidemargin}{0.35in}
\setlength{\evensidemargin}{0in}


\begin{document}


\section*{\S 問題}

\bigskip
\noindent {\bf [問題1]}~
(17)式を導け。(ヒント: $zp(z)$は奇関数なので
\begin{displaymath}
\int_{-\infty}^{\infty} zp(z){\rm d}z = 0
\end{displaymath}
となることを用いよ。)

\bigskip

\noindent {\bf [問題2]}~
銅線に電流$I$を流し、電圧$V$を測
定してこの銅線の電気抵抗$R$を求めようとした。このとき熱起電力$V_0$のため、電圧は
$V=RI+V_0$となる。流した電流を$I_i$、測定した電圧を$V_i(i=1, 2, \cdots,
n)$として、最小二乗法により$R$と$V_0$を求める式を導け。


\bigskip

\noindent {\bf [問題3]}~長方形板の寸法が縦400.0~[mm]、横500.0~[mm]、厚さ
10.0~[mm]であり、誤差が縦$\pm$0.5~[mm]、横$\pm$0.6~[mm]、厚さ
$\pm$0.1~[mm]であるとする。この板の体積の相対誤差を求めよ 。

\bigskip

\noindent {\bf [問題4]}~銅線の電気抵抗を測定して銅の比抵抗を求める。銅線
の直径を$D$、長さを$L$、電流を$I$、電位差を$V$とすれば、比抵抗は
\begin{displaymath}
\rho_{\rm r} = {\pi VD^2 \over 4IL}
\end{displaymath}
となる。$\rho_{\rm r}$の相対誤差を求め、銅線の直径を他の量より高い精度
で測定しなければいけない理由を考えよ。

\bigskip

\noindent {\bf [問題5]}~$x_1=200.01$、$x_2=1000$、$x_3=0.19994$のとき、
$Y=x_1 - x_2x_3$を以下の方針に沿って計算せよ。ただし、$x_1$と$x_3$は測定
値、$x_2$は確定値とする。

\begin{itemize}

\item[(1)]丸めないで

\item[(2)]4桁に丸めて

\item[(3)]3桁に丸めて

\end{itemize}


\noindent {\bf [問題6]}~数値はすべて測定値で、有効数字を表しているとして、
次式を計算せよ。計算は筆算で行い、途中の計算結果も明示すること。

\bigskip

(1) $152.3 + 6.478$  \hspace{1cm} (2) $58.36 \times 8.254$  
\hspace{1cm} (3) $8.472 \div 22.63$

\bigskip

\noindent {\bf [問題7]}~針金の直径を測って下表のようなデータを得た。最確
値を求め、残差から{\bf 付録5}の(33)式を用いて母集団の標準偏差を推定し、
確率誤差を求めよ。また、(18)式から試料平均の標準偏差を求めよ。

\begin{center}

表1. 針金の直径

\vspace{0.2cm}
\begin{tabular}{|c|c|c|c|c|c|} \hline
直径~[mm] & 1.014 & 1.016 & 1.011 & 1.017 & 1.022  \\  
\hline
\end{tabular}
\end{center}

\newpage

\noindent {\bf [問題8]}~重力加速度を測定するにあたり、おもりのつい
た紐の長さと振動周期をそれぞれ5回ずつ測定した。下表の測定値
について、まず紐の長さと振動周期について解析を行い、最確値、
母集団の標準偏差、試料平均の標準偏差値を求めよ({\bf 6.3}の
{\bf [例]}を参考にすること)。これから(21)、(22)式に基づき、
重力加速度の最確値と母集団の標準偏差および試料平均の標準偏差
を求めよ。

\begin{center}

表2. 紐の長さと振動周期

\vspace{0.2cm}
\begin{tabular}{|c|c|} \hline
$L$~[mm]  \\ \hline
1041      \\ \hline
1042       \\ \hline
1040       \\ \hline
1041       \\ \hline
1043       \\ \hline
\end{tabular}\ \ \ \ \ 
\begin{tabular}{|c|c|} \hline
 $T$~[s] \\ \hline
 2.070    \\ \hline
 2.040    \\ \hline
 2.055    \\ \hline
 2.045    \\ \hline
 2.035    \\ \hline
\end{tabular}
\end{center}


%\newpage

\noindent {\bf [問題9]}~{\bf 問題2}の結果を用いて下表の測定値より$R$と
$V_0$の最確値を求めよ。

\begin{center}

表3. 導線の電流電圧特性

\vspace{0.2cm}
\begin{tabular}{|c|c|} \hline
電流$I$~[mA] & 電圧$V$~[mV] \\ \hline
~99.40   & ~33.82    \\ \hline
198.12   & ~67.56    \\ \hline
301.50   & 102.72    \\ \hline
473.50   & 161.72    \\ \hline
\end{tabular}

\end{center}


\bigskip

\noindent {\bf [問題10]}~下表は極低温(温度$T$)における銅の比熱$C$の測定デー
タである。$y=C/T$、$x=T^2$を計算し、縦軸に$y$、横軸に$x$をとっ
たグラフを方眼紙に描け。比熱を $C=\gamma T + AT^3$と表したと
き、これを$x$と$y$の関係として表し、
$\gamma$と$A$の値を最小二乗法で決定せよ。

\begin{center}

表4. 極低温域における銅の比熱

\vspace{0.2cm}
\begin{tabular}{||c|c||c|c||} \hline
温度$T$~[K] & 比熱$C$~[mJ/mol$\cdot$K] & 温度$T$~[K] & 比 
熱$C$~[mJ/mol$\cdot$K]\\ \hline
4.12   &  6.017  &  3.01  &  3.391 \\ \hline
3.88   &  5.702  &  2.85  &  3.083 \\ \hline
3.80   &  5.148  &  2.45  &  2.479 \\ \hline
3.67   &  4.914  &  2.24  &  2.106 \\ \hline
3.52   &  4.710  &  2.01  &  1.760 \\ \hline
3.33   &  4.021  &  1.73  &  1.396 \\ \hline
3.16   &  3.820  &  1.36  &  1.042 \\ \hline
\end{tabular}

\end{center}

\bigskip

\noindent {\bf [問題11]}~等間隔で目盛ってある方眼紙を用いて$1 \sim 200$
までの片対数グラフを作成せよ。$1 \sim10$および$10 \sim 100$の間隔は
100~[mm]とし、$10 \sim 100$の間を間隔10ごとに目盛れ。

\bigskip

\newpage

\noindent {\bf [問題12]}~{\bf 問題11}で作ったグラフに、下表のデータを書き
入れ、$x$と$y$の関係を示せ。傾きはグラフから求めたものでよく、誤差処理は
行わなくてよい。必ず$y=\cdots$と書くこと。$\log y=\cdots$は不可。

\begin{center}

表5. ある材料の電圧電流特性

\vspace{0.2cm}
\begin{tabular}{||c|c||c|c||} \hline
電圧$x$~[V] & 電流$y$~[A] & 電圧$x$~[V] & 電流$y$~[A] \\  
\hline
0.45   &  ~3.02  &  2.95  &  ~30.2 \\ \hline
1.00   &  ~5.02  &  3.50  &  ~50.2 \\ \hline
1.50   &  ~7.96  &  4.00  &  ~79.6 \\ \hline
1.75   &  10.0~  &  5.00  &  200~  \\ \hline
2.50   &  20.0~  &        &        \\ \hline
\end{tabular}

\end{center}

\bigskip

\noindent {\bf [問題13]}~下表の測定データを市販の両対数グラフに書き入れ、
{\bf 問題12}と同様に$x$と$y$の関係を示せ。傾きはグラフから求めたものでよ
く、誤差処理は行わなくてよい。必ず $y=\cdots$という形で示すこと。

\begin{center}

表6. 時間と距離の関係

\vspace{0.2cm}
\begin{tabular}{||c|c||c|c||} \hline
$x$~[s] & $y$~[m] & $x$~[s] & $y$~[m] \\ \hline
1.0   &  ~4.90  &  ~4.0  &  ~78.4 \\ \hline
2.0   &  19.6~  &  ~5.0  &  122.5 \\ \hline
3.0   &  44.1~  &  10.0  &  490~  \\ \hline
\end{tabular}

\end{center}

\end{document}
